\documentclass[a4paper,12pt]{report}
\usepackage[utf8]{inputenc}
\usepackage{graphicx}
\usepackage{fancyhdr}
\usepackage{graphics}
\usepackage{hyperref}
\usepackage{capt-of}
\usepackage[utf8]{inputenc}
% \usepackage{arevmath}
\usepackage[noend]{algpseudocode}
\usepackage{algorithm}



\title{ETERNITY: FUNCTION}
\author{Avneet Kaur Pannu}
\date{July 2022}

\makeatletter
\let\thetitle\@title
\let\theauthor\@author
\let\thedate\@date
\makeatother

\pagestyle{fancy}
\fancyhf{}
\rhead{\thetitle}
\cfoot{\thepage}

\begin{document}

\begin{titlepage}
	\centering
    \vspace*{0.5 cm}
\begin{center}    \textsc{\Large Concordia University}\\[2.0 cm]	\end{center}
	\textsc{\Large  SOEN 6011 - Software Engineering Process }\\[0.5 cm]
	\rule{\linewidth}{0.2 mm} \\[0.4 cm]
	{ \huge \textbf \thetitle}\\[0.2 cm]
	{ \huge \textbf{$ab^{x}$}}
	\rule{\linewidth}{0.2 mm} \\[1.5 cm]

\begin{center}   {\Large Deliverable 1}\\[2.0 cm]
\end{center}

\begin{center}   {\Large \textbf{\theauthor}} \\[0.2 cm]
                 {\large Student ID : 40168576 }\\[0.2 cm]
                 {\large https://github.com/avneet-kaur/SOEN-6011}
\end{center}
\end{titlepage}

\renewcommand{\thesection}{\arabic{section}}
\tableofcontents
\pagebreak



\section{Introduction}
An exponential function is a function with the general form $ab^{x}$, $a\neq 0$, b is a positive real number and $b\neq 1$ .In an exponential function, a is constant, the base b is a constant, and the exponent x is a real variable.\cite{b1}

\subsection{Domain}
\begin{itemize}
\item The domain is all real numbers.
\\$- \infty < x < + \infty$, $x \in R$ \cite{b1}
\end{itemize}

\subsection{Co-Domain}
\begin{itemize}
  \item  The co-domain is also set of all real numbers.
\end{itemize}

\subsection{Characteristic}
\begin{itemize}


  \item \textbf{Exponential growth} : In the function  f(x) = $b^{x}$ when  $b > 1$, the function represents exponential growth. In figure 1, it is evident on the left side. \cite{b3}

  \item \textbf{Exponential decay} : In the function  f(x) = $b^{x}$ when  $0 < b < 1$, the function represents exponential decay. In figure 1, it is evident on the right side.\cite{b3}

   \item \textbf{Commutativity}:  Exponential function is not commutative which means $x^y \ne y^x $ for $x\ne y$. For example, $0^1 = 0$ and $1^0 = 1$.

   \item \textbf{Natural Exponential Function}: When the base is chosen to be b=e, the function f(x) = $e^x$ is called natural exponential function.\cite{b1}

   \item \textbf{} In the function f(x) = a$b^{x}$  when $|a| > 1$, it increases the speed of either growth or decay, and $0<|a|<1$ decreases the speed of either growth or decay.\cite{b2}

\begin{figure}
\includegraphics[width=15cm]{ExponentialGrowthandDecay.png}
\caption{Exponential Growth and Exponential Decay}
\label{exp}
\end{figure}



\end{itemize}

\section{Functional Requirement}
\subsection{Definitions and abbreviations}
\begin{center}
    \begin{tabular}{|c|c|}
         \hline
         Term & Definition \\
         \hline\hline
         FR & Functional Requirement \\
         \hline
         NFR & Non-Functional Requirement \\
         \hline
         User & End user are the human users who interacts with the system \\
         \hline
         System & Application which is used for solving exponential function. \\
         \hline
        \end{tabular}
        \label{tab:xyz}
         \captionof{table}{Definitions and abbreviations}
\end{center}

\subsection{Assumptions}
\begin{itemize}
    \item The calculator must accepts the exponential constant like e in addition to the constants a and b.
    \item

\end{itemize}

\subsection{Requirements}



\subsubsection{Functional Requirements}
\begin{itemize}
    \item
    \textbf{ID } \hspace{3cm} :FR1  \\
	\textbf{Type } \hspace{2.4cm}  :Functional\\
	\textbf{Version Number} \hspace{0.3cm} :1.0  \\
	\textbf{Owner } \hspace{1.98cm} : Avneet \\
	\textbf{Priority } \hspace{1.75cm} : High\\
	\textbf{Difficulty } \hspace{1.5cm} : Easy \\
	\textbf{Description }\hspace{1.2cm} :The calculator should ask the user to input a, b, and x. \\
	\textbf{Rationale }\hspace{1.6cm} :In order to process function f(x) = a$b^x$ and give output system needs input form the user. \\

	\item
    \textbf{ID } \hspace{3cm} :FR2  \\
	\textbf{Type } \hspace{2.4cm}  :Functional\\
	\textbf{Version Number} \hspace{0.3cm} :1.0  \\
	\textbf{Owner } \hspace{1.98cm} : Avneet \\
	\textbf{Priority } \hspace{1.75cm} : High\\
	\textbf{Difficulty } \hspace{1.5cm} : Easy\\
	\textbf{Description }\hspace{1.2cm} :When a user input is not a number, the system should provide an error message. \\
	\textbf{Rationale }\hspace{1.6cm} :The only acceptable input for an exponential function calculation is a number. \\

	\item
    \textbf{ID } \hspace{3cm} :FR3  \\
	\textbf{Type } \hspace{2.4cm}  :Functional\\
	\textbf{Version Number} \hspace{0.3cm} :1.0  \\
	\textbf{Owner } \hspace{1.98cm} : Avneet \\
	\textbf{Priority } \hspace{1.75cm} : High\\
	\textbf{Difficulty } \hspace{1.5cm} : Easy\\
	\textbf{Description }\hspace{1.2cm} :If a user enters incorrect data, the system shouldn't shut down but rather prompt users to reenter their data.\\
	\textbf{Rationale }\hspace{1.6cm} :The ability to perform calculations again and without closing the programme should be available to the user. \\

	\item
    \textbf{ID } \hspace{3cm} :FR4  \\
	\textbf{Type } \hspace{2.4cm}  :Functional\\
	\textbf{Version Number} \hspace{0.3cm} :1.0  \\
	\textbf{Owner } \hspace{1.98cm} : Avneet \\
	\textbf{Priority } \hspace{1.75cm} : High\\
	\textbf{Difficulty } \hspace{1.5cm} : Easy\\
	\textbf{Description }\hspace{1.2cm} : Whole numbers and rational numbers are accepted as user inputs.\\
	\textbf{Rationale }\hspace{1.6cm} : The code does not handle irrational numbers. For instance, $\pi$, $\sqrt{2}$ \\

	\item
    \textbf{ID } \hspace{3cm} :FR5  \\
	\textbf{Type } \hspace{2.4cm}  :Functional\\
	\textbf{Version Number} \hspace{0.3cm} :1.0  \\
	\textbf{Owner } \hspace{1.98cm} : Avneet \\
	\textbf{Priority } \hspace{1.75cm} : High\\
	\textbf{Difficulty } \hspace{1.5cm} : Easy\\
	\textbf{Description }\hspace{1.2cm} : Fractional inputs must be entered as double values.\\
	\textbf{Rationale }\hspace{1.6cm} : If a user wants to provide a base or exponent value of 1/2, they must do so as 0.5.\\

	\item
    \textbf{ID } \hspace{3cm} :FR6  \\
	\textbf{Type } \hspace{2.4cm}  :Functional\\
	\textbf{Version Number} \hspace{0.3cm} :1.0  \\
	\textbf{Owner } \hspace{1.98cm} : Avneet \\
	\textbf{Priority } \hspace{1.75cm} : High\\
	\textbf{Difficulty } \hspace{1.5cm} : Easy\\
	\textbf{Description }\hspace{1.2cm} : Base b is restricted to positive number.\\
	\textbf{Rationale }\hspace{1.6cm} : In order to guarantee $b^x$ is real number.\\


	\item
    \textbf{ID } \hspace{3cm} :FR7  \\
	\textbf{Type } \hspace{2.4cm}  :Functional\\
	\textbf{Version Number} \hspace{0.3cm} :1.0  \\
	\textbf{Owner } \hspace{1.98cm} : Avneet \\
	\textbf{Priority } \hspace{1.75cm} : High\\
	\textbf{Difficulty } \hspace{1.5cm} : Easy\\
	\textbf{Description }\hspace{1.2cm} : When any base value of b is raised to the power of x=0, the function's $b^x$ portion must return the value 1.\\
	\textbf{Rationale }\hspace{1.6cm} : For instance: 11 raised to the power 0 gives 1.\\

	\item
    \textbf{ID } \hspace{3cm} :FR8  \\
	\textbf{Type } \hspace{2.4cm}  :Functional\\
	\textbf{Version Number} \hspace{0.3cm} :1.0  \\
	\textbf{Owner } \hspace{1.98cm} : Avneet \\
	\textbf{Priority } \hspace{1.75cm} : High\\
	\textbf{Difficulty } \hspace{1.5cm} : Easy\\
	\textbf{Description }\hspace{1.2cm} : When base value b=0 is raised to any exponent value, the $b^x$ portion of the function must return 0.\\
	\textbf{Rationale }\hspace{1.6cm} : For instance, 0 raised to the power 11 yields 0.\\

\end{itemize}
\subsubsection{Non-Functional Requirements}
\begin{itemize}

	\item
    \textbf{ID } \hspace{3cm} :NFR1  \\
	\textbf{Type } \hspace{2.4cm}  :Non-Functional\\
	\textbf{Version Number} \hspace{0.3cm} :1.0  \\
	\textbf{Owner } \hspace{1.98cm} : Avneet \\
	\textbf{Priority } \hspace{1.75cm} : High\\
	\textbf{Difficulty } \hspace{1.5cm} : Easy\\
	\textbf{Description }\hspace{1.2cm} :An error message should be informative and relevant to the user.\\
	\textbf{Rationale }\hspace{1.6cm} : The user should be able to resolve simple problems on their own by understanding error message to enhance usability.\\

	\item
    \textbf{ID } \hspace{3cm} :NFR2  \\
	\textbf{Type } \hspace{2.4cm}  :Non-Functional\\
	\textbf{Version Number} \hspace{0.3cm} :1.0  \\
	\textbf{Owner } \hspace{1.98cm} : Avneet \\
	\textbf{Priority } \hspace{1.75cm} : High\\
	\textbf{Difficulty } \hspace{1.5cm} : Easy\\
	\textbf{Description }\hspace{1.2cm} :The command line interface ought to be user-friendly.\\
	\textbf{Rationale }\hspace{1.6cm} : The system should be simple for the user to operate. \\

	\item
    \textbf{ID } \hspace{3cm} :NFR3  \\
	\textbf{Type } \hspace{2.4cm}  :Non-Functional\\
	\textbf{Version Number} \hspace{0.3cm} :1.0  \\
	\textbf{Owner } \hspace{1.98cm} : Avneet \\
	\textbf{Priority } \hspace{1.75cm} : High\\
	\textbf{Difficulty } \hspace{1.5cm} : Easy\\
	\textbf{Description }\hspace{1.2cm} :The outcome must be accurate.\\
	\textbf{Rationale }\hspace{1.6cm} :  To enhance the accuracy of system. It is inappropriate to display incorrect ouptut to the user.\\

	\item
    \textbf{ID } \hspace{3cm} :NFR4  \\
	\textbf{Type } \hspace{2.4cm}  :Non-Functional\\
	\textbf{Version Number} \hspace{0.3cm} :1.0  \\
	\textbf{Owner } \hspace{1.98cm} : Avneet \\
	\textbf{Priority } \hspace{1.75cm} : High\\
	\textbf{Difficulty } \hspace{1.5cm} : Easy\\
	\textbf{Description }\hspace{1.2cm} :There should be no more than 5 seconds of calculation time.\\
	\textbf{Rationale }\hspace{1.6cm} :  In order to improve the performance of the system.\\

	\item
    \textbf{ID } \hspace{3cm} :NFR5  \\
	\textbf{Type } \hspace{2.4cm}  :Non-Functional\\
	\textbf{Version Number} \hspace{0.3cm} :1.0  \\
	\textbf{Owner } \hspace{1.98cm} : Avneet \\
	\textbf{Priority } \hspace{1.75cm} : High\\
	\textbf{Difficulty } \hspace{1.5cm} : Easy\\
	\textbf{Description }\hspace{1.2cm} : System should be maintainable for the duration of its anticipated lifetime and able to accommodate new requirements in response to stakeholders' changing needs. \\
	\textbf{Rationale }\hspace{1.6cm} : Since future changes to software systems are inevitable. Maintainable systems are therefore simpler to alter.\\

	\item
    \textbf{ID } \hspace{3cm} :NFR6  \\
	\textbf{Type } \hspace{2.4cm}  :Non-Functional\\
	\textbf{Version Number} \hspace{0.3cm} :1.0  \\
	\textbf{Owner } \hspace{1.98cm} : Avneet \\
	\textbf{Priority } \hspace{1.75cm} : High\\
	\textbf{Difficulty } \hspace{1.5cm} : Easy\\
	\textbf{Description }\hspace{1.2cm} : The system should be developed using widely used and standardised language. \\
	\textbf{Rationale }\hspace{1.6cm} : Java is used to build a system which is platform independent hence make the system portable.\\


\end{itemize}
\pagebreak

\section{Algorithm}

\subsection{Pseudocode}

\begin{algorithm}
\caption{Iterative Algorithm to calculate: $ab^x$ }
\begin{algorithmic}
\Procedure{$calculateExponentialFunction$}{$a,b,x$}
\State \textbf{input: } String a, b, x
\State \textbf{output: } double res
\State  res = 1
\State temp = 1
\If{((a $||$ b) == "0")}
    \Return res
\Else
    \If{(b == "e")}
    \State exposum = 1
    \State nterms = 25
    \State for i $<=$ nterms
        \State exposum = 1+x * exposum/i
    \State end
\State \Return a*exposum
    \Else
        \State for temp $<=$x
            \State res = res * b
            \State temp = temp + 1
        \State end
        \State \Return a*res
        \EndIf
\EndIf
\EndProcedure
\end{algorithmic}
\end{algorithm}

% Second logic: recursive approach
\begin{algorithm}
\caption{Recursive Algorithm to calculate: $ab^x$ }
\begin{algorithmic}
\Procedure{$calculateExponentialFunction$}{$a,b,x$}
\State \textbf{input: } string a, b, x
\State \textbf{output: } double res
\State  res=0
\If{(a $||$ b == 0)}
    \State \Return res
\ElsIf{(b=="e")}
    \State res = naturalExponential(x)
\Else
 \State res = calculatePower(b,x)
\EndIf
\State res = $a * res$
\State \Return res
\EndProcedure
\\
\Procedure{$naturalExponential$}{$x$}
\State \textbf{input: } int x
\State \textbf{output: } double exposum
\State nterms = 25
\State exposum = 1
\State for i $<=$ nterms
    \State exposum = 1+x * exposum/i
    \State end
\State \Return exposum
\EndProcedure
\\
\Procedure{$calculatePower$}{$b,x$}
\State \textbf{input: } double b, int x
\State \textbf{output: } double res
\If{(($x < 0$)}
\State \Return $1.0/powHelper(b,x)$
   \EndIf
\State \Return $powHelper(b,x)$
\EndProcedure
\\
\Procedure{$powHelper$}{$b,x$}
\State \textbf{input: } double b, int x
\State \textbf{output: } double res
\If{($x == 0$)}
    \Return 1
\EndIf
\If{($x == 1$)}
    \Return b
\EndIf
\If{($x mod 2$ == 0)}
\State \Return $powerHandler(b*b,  n/2)$
\Else
\State \Return $b * powerHandler(b*b,  n/2)$
\EndIf
\EndProcedure
\end{algorithmic}
% \end{algorithm}

\end{algorithm}

\pagebreak

\subsection{Description}

\subsubsection{Algorithm1}

\textbf{ Description: }
\\\textbf{ Rationale: }
\\\textbf{ Complexity: }
\\\textbf{ Advantages: }
\\\textbf{ Disadvantages: }


\subsubsection{Algorithm2}

\textbf{ Description: }
\\\textbf{ Rationale: }
\\\textbf{ Complexity: }
\\\textbf{ Advantages: }
\\\textbf{ Disadvantages: }

\subsection{Mindmap for Pseudocode format Selection}




\begin{thebibliography}{9}
\bibitem{b1} Zill, D. G., \& Wright, W. S. (2011). Calculus: Early transcendentals. Jones and Bartlett Publishers.
\bibitem{b2} Rock, N. M. (2007). Standards driven math. student standards handbook. Team Rock Press.
\bibitem{b3}Exponential growth and decay - virtuallearningacademy.net. (n.d.). Retrieved August 3, 2022, from  \url{https://virtuallearningacademy.net/VLA/LessonDisplay/Lesson6156/MATHALGIIBU17Exponential_Decay.pdf}
\bibitem{b4}GeeksforGeeks. (2022, July 19). Write a program to calculate POW(X,N). GeeksforGeeks. Retrieved August 5, 2022, from https://www.geeksforgeeks.org/write-a-c-program-to-calculate-powxn/
\bibitem{b5}Exponential Function Calculator. High accuracy calculation for life or science. (n.d.). Retrieved August 5, 2022, from https://keisan.casio.com/exec/system/1223447896
\end{thebibliography}
\end{document}
